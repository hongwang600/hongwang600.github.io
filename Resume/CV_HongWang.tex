% !TEX TS-program = xelatex
% !TEX encoding = UTF-8 Unicode
% -*- coding: UTF-8; -*-
% vim: set fenc=utf-8

%%%%%%%%%%%%%%%%%%%%%%%%%%%%%%%%%%%%%%%%%%%%%%%%%%%%%%%%%%%%%%%%%
%% SIMPLE-RESUME-CV
%% <https://github.com/zachscrivena/simple-resume-cv>
%% This is free and unencumbered software released into the
%% public domain; see <http://unlicense.org> for details.
%%%%%%%%%%%%%%%%%%%%%%%%%%%%%%%%%%%%%%%%%%%%%%%%%%%%%%%%%%%%%%%%%

% See "README.md" for instructions on compiling this document.

\documentclass[letterpaper,MMMyyyy,nonstopmode]{simpleresumecv}
% Class options:
% a4paper, letterpaper, nonstopmode, draftmode
% MMMyyyy, ddMMMyyyy, MMMMyyyy, ddMMMMyyyy, yyyyMMdd, yyyyMM, yyyy

%%%%%%%%%%%%%%%%%%%%%%%%%%%%%%%%%%%%%%%%%%%%%%%%%%%%%%%%%%%%%%%%%
%% PREAMBLE.
%%%%%%%%%%%%%%%%%%%%%%%%%%%%%%%%%%%%%%%%%%%%%%%%%%%%%%%%%%%%%%%%%

% CV Info (to be customized).
\newcommand{\CVAuthor}{Hong Wang}
\newcommand{\CVTitle}{Hong Wang's CV}
\newcommand{\CVNote}{Last updated on {\today}}
\newcommand{\CVWebpage}{http://hongwang600.github.io}

\usepackage{enumitem}

% PDF settings and properties.
\hypersetup{
pdftitle={\CVTitle},
pdfauthor={\CVAuthor},
pdfsubject={\CVWebpage},
pdfcreator={XeLaTeX},
pdfproducer={},
pdfkeywords={},
unicode=true,
bookmarks=true,
bookmarksopen=true,
pdfstartview=FitH,
pdfpagelayout=OneColumn,
pdfpagemode=UseOutlines,
hidelinks,
breaklinks}

% Shorthand.
\newcommand{\Code}[1]{\mbox{\textbf{#1}}}
\newcommand{\CodeCommand}[1]{\mbox{\textbf{\textbackslash{#1}}}}

%%%%%%%%%%%%%%%%%%%%%%%%%%%%%%%%%%%%%%%%%%%%%%%%%%%%%%%%%%%%%%%%%
%% ACTUAL DOCUMENT.
%%%%%%%%%%%%%%%%%%%%%%%%%%%%%%%%%%%%%%%%%%%%%%%%%%%%%%%%%%%%%%%%%

\begin{document}

%%%%%%%%%%%%%%%
% TITLE BLOCK %
%%%%%%%%%%%%%%%

\Title{\CVAuthor}

\begin{SubTitle}
\href{https://www.google.com/maps/place/Phelps+Hall/@34.4161308,-119.8468366,17z/data=!3m1!4b1!4m5!3m4!1s0x80e93f7042b51aa9:0xc80cbd08cacd9e49!8m2!3d34.4161308!4d-119.8446426}
{3530 Phelps Hall, UC Santa Barbara, California, 93106}
\par
\href{mailto:hongwang600@ucsb.edu}
{hongwang600@ucsb.edu}
\,\SubBulletSymbol\,
\href{\CVWebpage}
{\url{\CVWebpage}}
\,\SubBulletSymbol\,
\href{https://scholar.google.com/citations?user=M9uQHIUAAAAJ&hl=en&authuser=1#}
{Google Scholar}
\end{SubTitle}

\begin{Body}

%%%%%%%%%%%%%%%
%% RESEARCH INTEREST %%
%%%%%%%%%%%%%%%

% \Section
% {Research \newline Interests}
% {Research Interests}
% {PDF:ResearchInterst}

% \Entry
% I am a fifth year CS Ph.D. candidate at UC Santa Barbara, advised by Prof. Xifeng Yan. My current research focuses on building task-oriented dialogue system under low-resource scenarios. I am also interested in open-domain question answering, pre-training methods with self-supervised learning, information extraction and reasoning over knowledge graphs, especially for lifelong learning and few-shot learning settings, and low resource Natural Language Processing such as transferring knowledge from rich resource languages to other low resource languages.

%%%%%%%%%%%%%%%
%% SELF INTRODUCTION%%
%%%%%%%%%%%%%%%

\Section
{About me}
{About-me}
{PDF:AboutMe}

\Entry
I'm an Applied Scientist at Amazon AWS Connect. I obtained my Ph.D. in Computer Science at UC Santa Barbara. My research mainly focuses on Machine Learning and NLP, especially on the dialogue system, question answering and information retrieval. As for industrial experience, I have interned as Applied Scientist and Research Scientist at Amazon Alexa, Meta AI and Microsoft Research, where I accumulated much hands-on experience for large-scale data preparation, model design, training, evaluation and deployment. In addition, I have led a team of 5 PhD talents to participate the fierce Amazon Taskbot Challenge , competing with 10 university teams from 3 continents and successfully getting into the final round. Personally, I am always curious and passionate about applying ML to solve complex problems in an applied environment! 


%%%%%%%%%%%%%%%
%% EDUCATION %%
%%%%%%%%%%%%%%%

\Section
{Education}
{Education}
{PDF:Education}

\Entry
\textbf{University of California, Santa Barbara},
Santa Barbara, California, USA

\Gap
\BulletItem
Ph.D. in Computer Science
\hfill
\DatestampYMD{2018}{09}{01} -- \DatestampYMD{2023}{06}{2}
\begin{Detail}
\SubBulletItem
Advisor:
Prof.~Xifeng~Yan
\SubBulletItem
Focus:
Natural Language Processing, Deep Learning
\SubBulletItem
Chancellor's Fellowship for three years
\end{Detail}

\BigGap
\Entry
\textbf{Nanjing University},
Nanjing, Jiangsu, China

\Gap
\BulletItem
B.S. in Computer Science
\hfill
\DatestampYMD{2014}{09}{01} --
\DatestampYMD{2018}{06}{30}
\begin{Detail}
% \SubBulletItem
% Thesis Advisor: Prof.~Yang~Yu
% \SubBulletItem
% Focus: Derivative-free Optimization
\SubBulletItem
Research assistant in Learning And Mining from DatA (LAMDA) Group
\SubBulletItem
Excellent Bachelor Thesis Award
\SubBulletItem
GPA Ranking 4/150

\end{Detail}

%%%%%%%%%%%%%%%%%%%%%%%%%
%% RESEARCH EXPERIENCE %%
%%%%%%%%%%%%%%%%%%%%%%%%%

\Section
{Working Experience}
{Experience}
{PDF:Experience}

\BigGap
\Entry
\textbf{Amazon Alexa},
Wakeword Group, Cambridge, MA
\BulletItem
Applied Scientist Intern
\hfill
\DatestampYMD{2022}{06}{13} --
\DatestampYMD{2022}{09}{16}
\begin{Detail}
\SubBulletItem
Supervisor:
Dr.~Gengshen~Fu, Rajath Kumar
\SubBulletItem
Project: Unified Wakeword and compact spoken language understanding under low-resource scenarios
\SubBulletItem
Applied data synthesis approach to generate audio data for low-resource intents.
\SubBulletItem
Explored data augmentation techniques to improve generation from low-resource to rich-resource intents.
\SubBulletItem
Proposed techniques to further improve generalization such as using separate classification layers and conducting alignment between synthetic and real data.
\SubBulletItem
Submitted a paper and delivered a demo for unified wakeword and intent detection on device.

\end{Detail}

\BigGap
\Entry
\textbf{Amazon Alexa},
Wakeword Group, Cambridge, MA
\BulletItem
Applied Scientist Intern
\hfill
\DatestampYMD{2021}{06}{14} --
\DatestampYMD{2021}{09}{17}
\begin{Detail}
\SubBulletItem
Supervisor:
Dr.~Gengshen~Fu, Dr.~Bo~Xiao
\SubBulletItem
Project:
Unified Wakeword and intent detection on device
\SubBulletItem
Designed model architecture for unified Wakeword and intent detection.
\SubBulletItem
Proposed new loss function to reduce false accepts for the streaming model.
\SubBulletItem
Supported entity recognition for a finite set, which largely increases the coverage of NLU on device.
\SubBulletItem
Achieved a promising coverage for intent and entity recognition when matching the precision as cloud model.
\SubBulletItem
Submitted a patent and a paper.
% \SubBulletItem
% Description:
% Existing workflow typically detect wakeword on device and then send audio for further processing on cloud. In this project, we explored the possibility of detecting some common intents together with the wakeword using a unified model on device. 
\end{Detail}

\BigGap
\Entry
\textbf{Meta AI},
Language and Translation Technologies (LATTE) Group, Menlo Park, CA
\BulletItem
Research Intern
\hfill
\DatestampYMD{2020}{06}{15} --
\DatestampYMD{2020}{09}{18}
\begin{Detail}
\SubBulletItem
Supervisor:
Dr.~Ves~Stoyanov
\SubBulletItem
Project:
Memory augmented pretrained language model
\SubBulletItem
Implemented product key memory (PKM) in Fairseq framework for language modeling, achieving better perplexity during pre-training while using similar computation cost.
% \SubBulletItem
% Description:
% Language models can achieve better performance when increasing the model size. However, the memory requirement and computational cost would also increase as the model size. In this project, I explored product key memory (PKM) for pretrained model on language modeling task. It shows that language models can achieve lower perplexity when augmented with PKM using similar computational cost.
\end{Detail}

\BigGap
\Entry
\textbf{Microsoft Research Asia},
Machine Learning Group, Beijing, China
\BulletItem
Software Engineer Intern
\hfill
\DatestampYMD{2017}{10}{15} --
\DatestampYMD{2018}{01}{15}
\begin{Detail}
\SubBulletItem
Supervisor:
Dr.~Tao~Qin
% \SubBulletItem
% Project:
% Classification systems and models for ads relevance and click prediction
\SubBulletItem
Built classification models for ads relevance and click prediction.
\SubBulletItem
Investigated the rationality of data, extracted new features and designed new models.
\SubBulletItem
Explored how to alleviate position bias problem in advertisement.
% Description:
% My task is to build classification systems and models for ads relevance and click prediction. I investigated the rationality of data, extract new features, design new models and try to tackle the position bias Problem in advertisement.
\end{Detail}

% \BigGap
% \Entry
% \textbf{Learning And Mining from DatA (LAMDA) Group},
% Nanjing University
% \BulletItem
% Research Assistant
% \hfill
% \DatestampYMD{2016}{09}{15} --
% \DatestampYMD{2018}{06}{15}
% \begin{Detail}
% \SubBulletItem
% Supervisor:
% Prof.~Yang~Yu
% \SubBulletItem
% Project:
% A novel approach to improve derivative-free algorithms in noisy environments
% \end{Detail}

%%%%%%%%%%%%%%%%%%
%% PUBLICATIONS %%
%%%%%%%%%%%%%%%%%%
\newpage
\Section
{Skills}
{Skills}
{PDF:Skills}
\BulletItem
\textbf{Programming Language}: Python, C, C++

\BulletItem
\textbf{Machine Learning}: Tensorflow, Pytorch, Spark, Transformers, Fairseq, 
Dialogue System, Question Answering, Information Retrieval, Language Model Pretraining

\BulletItem
\textbf{Computer Science}: SQL, Compiler, Operating System, Distributed System,
Networking, Edge Computing, Database, Data Structure, Algorithms

% \BulletItem
% \textbf{Writing}: LaTex

\Section
{Publications}
{Publications}
{PDF:Publications}

% \SubSection
% {Journals}
% {Journals}
% {PDF:Journals}

% % Declare a new group to limit the scope of \MaxNumberedItem to this subsection.
% \begingroup
% \renewcommand{\MaxNumberedItem}{[88]}

% \Gap
% \NumberedItem{[10]}
% \href{http://www.example.com/my-paper-doi-5}
% {\underline{J.~Doe}, J.~Citizen, and A.~Yone,
% ``On lasers and climate change,''
% \textit{Journal of Science},
% vol.~89,
% no.~2,
% pp.~4123--4133,
% \DatestampYM{2008}{02}.}

% \Gap
% \NumberedItem{[1]}
% \href{http://www.example.com/my-paper-doi-4}
% {\underline{J.~Doe} and J.~Citizen,
% ``Measuring the extent of climate change,''
% \textit{Global Scientific Journal},
% vol.~12,
% no.~4,
% pp.~330--352,
% \DatestampYM{2006}{12}.}

% \endgroup

\BigGap
\SubSection
{Conferences}
{Conferences}
{PDF:Conferences}

% Declare a new group to limit the scope of \MaxNumberedItem to this subsection.
%\begingroup{}
%\renewcommand{\MaxNumberedItem}{[8]}
\begin{enumerate}[label={[\arabic*]~~}]
\item
Jing Qian$^*$, \underline{Hong Wang}$^*$, Zekun Li, Shiyang Li, Xifeng Yan,
``Limitations of Language Models in Arithmetic and Symbolic Induction'',
in \textit{Proceedings of the 61st Annual Meeting of the Association for Computational Linguistics (ACL 2023)}, Toronto, Canada, 
\DatestampYM{2023}{07}.

\item 
Zekun Li, Wenhu Chen, Shiyang Li, \underline{Hong Wang}, Jing Qian, Xifeng Yan,
``Dialogic: Controllable Dialogue Simulation with In-Context Learning'',
in \textit{Findings of the Conference on Empirical Methods in Natural Language Processing (Findings of EMNLP 2022)}, Abu Dhabi,
\DatestampYM{2022}{12}

\item
Zekun Li$^*$, \underline{Hong Wang}$^*$, Alon Albalak, Yingrui Yang, Jing Qian, Shiyang Li, Xifeng Yan,
``Making Something out of Nothing: Building Robust Task-oriented Dialogue Systems from Scratch'',
in \textit{Proceedings of Alexa Prize TaskBot},
\DatestampYM{2022}{06}.

\item
Jing Qian, \underline{Hong Wang}, Mai ElSherief, Xifeng Yan, 
``Lifelong Learning of Hate Speech Classification on Social Media'',
in \textit{Proceedings of 2021 Annual Conference of the North American Chapter of the Association for Computational Linguistics (NAACL-HLT 2021)}, online, \DatestampYM{2021}{06}.

\item
\underline{Hong Wang}$^*$, Wenhan Xiong$^*$, William Yang Wang,
``Progressively Pretrained Dense Corpus Index for Open-Domain Question Answering'',
in \textit{Proceedings of the 16th conference of the European Chapter of the Association for Computational Linguistics (EACL 2021)}, online, 
\DatestampYM{2021}{04}.

\item
Wenhu Chen, Hanwen Zha, Zhiyu Chen, Wenhan Xiong, \underline{Hong Wang}, William Wang,
``HybridQA: A Dataset of Multi-Hop Question Answering over Tabular and Textual Data'',
In \textit{Findings of the Conference on Empirical Methods in Natural Language Processing (Findings of EMNLP 2020)}, online, 
\DatestampYM{2020}{11}.

\item
Wenhu Chen, Hongmin Wang, Jianshu Chen, Yunkai Zhang, \underline{Hong Wang}, Shiyang Li, Xiyou Zhou and William Yang Wang,
``TabFact: A Large-scale Dataset for Table-based Fact Verification'',
In \textit{Proceedings of the 8th International Conference on Learning Representations (ICLR 2020)}, Addis Ababa, Ethiopia,
\DatestampYM{2020}{04}.

\item
\underline{Hong~Wang}, Xin~Wang, Wenhan~Xiong, Mo~Yu, Xiaoxiao~Guo, Shiyu~Chang and William~Wang,
``Self-Supervised Learning for Contextualized Extractive Summarization'',
in \textit{Proceedings of the 57th Annual Meeting of the Association for Computational Linguistics (ACL 2019)},
Florence, Italy,
\DatestampYM{2019}{07}.

\item
Wenhan~Xiong, Jiawei~Wu, \underline{Hong~Wang}, Vivek~Kulkarni, Mo~Yu, Xiaoxiao~Guo, Shiyu~Chang and William~Wang,
``TweetQA: A Social Media Focused Question Answering Dataset'',
in \textit{Proceedings of the 57th Annual Meeting of the Association for Computational Linguistics (ACL 2019)},
Florence, Italy,
\DatestampYM{2019}{07}.

\item
\underline{Hong~Wang}, Wenhan~Xiong, Mo~Yu, Xiaoxiao~Guo, Shiyu~Chang and William~Wang,
``Sentence Embedding Alignment for Lifelong Relation Extraction'',
in \textit{Proceedings of the 17th Annual Conference of the North American Chapter of the Association for Computational Linguistics: Human Language Technologies (NAACL-HLT 2019)},
Minneapolis, MN, USA,
\DatestampYM{2019}{06}.

\item
\underline{Hong~Wang}, Hong~Qian and Yang~Yu,
``Noisy derivative-free optimization with value suppression'',
in \textit{Proceedings of the 32nd AAAI Conference on Artificial Intelligence (AAAI 2018)},
New Orleans, LA, USA,
\DatestampYM{2018}{02}.
\end{enumerate}

\BigGap
\SubSection
{Preprints}
{Preprints}
{PDF:Preprints}
\begin{enumerate}[label={[\arabic*]~~}]
\item
Shiyang Li, Jianshu Chen, Yelong Shen, Zhiyu Chen, Xinlu Zhang, Zekun Li, \underline{Hong Wang}, Jing Qian, Baolin Peng, Yi Mao, Wenhu Chen, Xifeng Yan,
``Explanations from Large Language Models Make Small Reasoners Better'',
technical report,
\DatestampYM{2022}{10}

\item
Wenhan~Xiong, Mo~Yu, Xiaoxiao~Guo, \underline{Hong~Wang}, Shiyu~Chang, Murray~Campbell and William~Wang,
``Simple yet Effective Bridge Reasoning for Open-Domain Multi-Hop Question Answering'',
MRQA: Machine Reading for Question Answering workshop at EMNLP-IJCNLP 2019, Hong Kong, China,
\DatestampYM{2019}{11}.

\item
\underline{Hong Wang}, Christfried Focke, Rob Sylvester, Nilesh Mishra and William Wang,
``Fine-tune Bert for DocRED with Two-step Process'',
technical report,
\DatestampYM{2019}{09}.

\item
\underline{Hong~Wang}, Wenhan~Xiong, Mo~Yu, Xiaoxiao~Guo, Shiyu~Chang and William~Wang,
``Meta Reasoning over Knowledge Graphs'',
technical report,
\DatestampYM{2019}{06}.
\end{enumerate}

%\endgroup

\Section
{Patents}
{Patens}
{PDF:Patens}

\begin{enumerate}[label={[\arabic*]~~}]
\item
\underline{Hong~Wang}, Gengshen~Fu, Bo~Xiao and Tao~Zhang,
``Unified wakeword and language understanding processing'', patent submitted,
\DatestampYM{2021}{09}.
\end{enumerate}


%%%%%%%%%%%%%%%%%%%%%%%%%%%
%% AWARDS & SCHOLARSHIPS %%
%%%%%%%%%%%%%%%%%%%%%%%%%%%

\Section
{Awards \&\newline
Honors}
{Awards \& Honors}
{PDF:AwardsAndHonors}

\Entry
\textbf{Chancellor's Fellowship}, University of California, Santa Barbara
\hfill
\DatestampYM{2022}{09}

\Entry
\textbf{Chancellor's Fellowship}, University of California, Santa Barbara
\hfill
\DatestampYM{2020}{09}

\Gap
\Entry
\textbf{Chancellor's Fellowship}, University of California, Santa Barbara
\hfill
\DatestampYM{2018}{09}

\Gap
\Entry
\textbf{Outstanding Undergraduate Award}, Nanjing University
\hfill
\DatestampYM{2018}{06}

\Gap
\Entry
\textbf{Excellent Bachelor Thesis Award}, Nanjing University
\hfill
\DatestampYM{2018}{06}

\Gap
\Entry
\textbf{National Scholarship}, Ministry of Education, China
\hfill
\DatestampY{2016} -- \DatestampY{2017}

\Gap
\Entry
\textbf{National Endeavor Scholarship}, Ministry of Education, China
\hfill
\DatestampY{2014} -- \DatestampY{2016}

\Section
{Service}
{Service}
{PDF:Service}

\SubSection
{Committee Member \& Reviewer}
{Committee Member \& Reviewer}
{PDF:Committee-Member/Reviewer}

    \BulletItem Conference on Empirical Methods in Natural Language Processing and 9th International Joint Conference on Natural Language Processing (EMNLP-IJCNLP-19), 2019
    \BulletItem The Thirty-Fourth AAAI Conference on Artificial Intelligence (AAAI-20), 2020
    \BulletItem The 58th annual meeting of the Association for Computational Linguistics (ACL-20), 2020
    \BulletItem The Conference on Empirical Methods in Natural Language Processing (EMNLP-20), 2020
    \BulletItem The 1st Conference of the Asia-Pacific Chapter of the Association for Computational Linguistics and the 10th International Joint Conference on Natural Language Processing (AACL-IJCNLP-20), 2020
    \BulletItem The Thirty-Fifth AAAI Conference on Artificial Intelligence (AAAI-21), 2021
    \BulletItem The 16th conference of the European Chapter of the Association for Computational Linguistics (EACL-21), 2021. 
    \BulletItem The Annual Conference of the North American Chapter of the Association for Computational Linguistics (NAACL-HLT-21), 2021
    \BulletItem The Joint Conference of the 59th Annual Meeting of the Association for Computational Linguistics and the 11th International Joint Conference on Natural Language Processing (ACL-IJCNLP 21), 2021
    \BulletItem The 10th CCF International Conference on Natural Language Processing and Chinese Computing (NLPCC2021), 2021
    \BulletItem The thirty-fifth Conference on Neural Information Processing Systems (NeurIPS 2021), 2021
    \BulletItem ACL Rolling Review November, 2021
    \BulletItem ACL Rolling Review December, 2021
    \BulletItem The Tenth International Conference on Learning Representations (ICLR 2022), 2022
    \BulletItem The 13th Edition of its Language Resources and Evaluation Conference (LREC 2022), 2022
    \BulletItem The 39th International Conference on Machine Learning (ICML 2022), 2022
    \BulletItem The thirty-sixth Conference on Neural Information Processing Systems (NeurIPS 2022), 2022
    \BulletItem The 2022 Conference on Empirical Methods in Natural Language Processing (EMNLP 2022), 2022
    \BulletItem The 17th Conference of the European Chapter of the Association for Computational Linguistics (EACL-23), 2023
    \BulletItem The 40th International Conference on Machine Learning (ICML 2023), 2023
    \BulletItem The 61st Annual Meeting of the Association for Computational Linguistics (ACL-23), 2023
    \BulletItem 29TH ACM SIGKDD Conference on Knowledge Discovery and Data Mining (KDD 2023)

\Section
{Teaching}
{Teaching}
{PDF:Teaching}
\BulletItem Teaching Assistant, CS165B, Machine Learning, Fall 2019, UCSB
\BulletItem Teaching Assistant, CS165B, Machine Learning, Spring 2020, UCSB
\BulletItem Teaching Assistant, CS165B, Machine Learning, Spring 2021, UCSB

\Section
{Talks \& \newline Presentations}
{Talks \& presentations}
{PDF:Invited-talks}

\Entry \textbf{Sentence Embedding Alignment for Lifelong Relation Extraction.} \hfill \DatestampYMD{2019}{05}{28}
\begin{Detail}
\BulletItem
IBM invited online talk
\end{Detail}


\end{Body}

%%%%%%%%%%%
% CV NOTE %
%%%%%%%%%%%

\UseNoteFont%
\null\hfill%
[\textit{\CVNote}]

\end{document}
